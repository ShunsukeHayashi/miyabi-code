\documentclass[11pt,a4paper]{article}
\usepackage[utf8]{inputenc}
\usepackage[T1]{fontenc}
\usepackage{xeCJK}
\setCJKmainfont{Hiragino Kaku Gothic Pro W3}
\usepackage{geometry}
\geometry{margin=2.5cm}
\usepackage{graphicx}
\usepackage{xcolor}
\usepackage{tikz}
\usepackage{tcolorbox}
\usepackage{booktabs}
\usepackage{longtable}
\usepackage{enumitem}
\usepackage{hyperref}
\usepackage{fancyhdr}
\usepackage{titlesec}

% カラー定義
\definecolor{primary}{RGB}{37, 99, 235}
\definecolor{success}{RGB}{34, 197, 94}
\definecolor{warning}{RGB}{245, 158, 11}
\definecolor{danger}{RGB}{239, 68, 68}
\definecolor{lightbg}{RGB}{248, 250, 252}

% ヘッダー設定
\pagestyle{fancy}
\fancyhf{}
\fancyhead[L]{\small 安全書類管理システム 事業計画書}
\fancyhead[R]{\small \thepage}
\renewcommand{\headrulewidth}{0.5pt}

% タイトル設定
\titleformat{\section}{\Large\bfseries\color{primary}}{}{0em}{}[\titlerule]
\titleformat{\subsection}{\large\bfseries}{}{0em}{}

% tcolorbox設定
\tcbuselibrary{skins,breakable}
\newtcolorbox{highlight}[1][]{
  colback=lightbg,
  colframe=primary,
  fonttitle=\bfseries,
  #1
}
\newtcolorbox{kpi}{
  colback=success!10,
  colframe=success,
  title=KPI,
  fonttitle=\bfseries
}

\begin{document}

% タイトルページ
\begin{titlepage}
\centering
\vspace*{2cm}

{\Huge\bfseries\color{primary} 安全書類管理システム\\[0.5cm]
事業計画書}

\vspace{1cm}

{\Large Lark Base プラットフォーム}

\vspace{2cm}

\begin{tcolorbox}[colback=lightbg,colframe=primary,width=12cm]
\centering
\begin{tabular}{ll}
\textbf{市場機会} & TAM 4,800億円 \\
\textbf{目標ARR} & 11.5億円(3年目) \\
\textbf{導入企業} & 3,000社(3年目) \\
\textbf{投資回収} & 1.1ヶ月 \\
\end{tabular}
\end{tcolorbox}

\vfill

{\large 作成日: 2025年11月21日}\\
{\large バージョン: 1.0}

\end{titlepage}

% 目次
\tableofcontents
\newpage

% エグゼクティブサマリー
\section{エグゼクティブサマリー}

\subsection{事業概要}

Lark Base プラットフォーム上に構築した、建設業向け統合安全書類管理システムを中小建設会社・専門工事業者向けに提供するSaaSビジネス。

\begin{kpi}
\begin{tabular}{lll}
\textbf{製品構成} & 37テーブル & 8サブシステム統合 \\
\textbf{自動化} & 29ビュー & 50+リンク関係 \\
\end{tabular}
\end{kpi}

\subsection{市場機会}

\begin{itemize}[leftmargin=*]
\item \textbf{TAM}: 4,800億円(建設業DX市場全体)
\item \textbf{SAM}: 960億円(安全管理・労務管理SaaS)
\item \textbf{SOM}: 48億円(中小企業向けセグメント)
\end{itemize}

\subsection{収益目標(3年後)}

\begin{center}
\begin{tabular}{|l|c|c|c|}
\hline
\textbf{指標} & \textbf{1年目} & \textbf{2年目} & \textbf{3年目} \\
\hline
導入企業数 & 300社 & 1,200社 & 3,000社 \\
ARR & 9,000万円 & 4.3億円 & 11.5億円 \\
営業利益率 & -30\% & 5\% & 20\% \\
\hline
\end{tabular}
\end{center}

\newpage

% システムアーキテクチャ
\section{システムアーキテクチャ}

\subsection{全体構成図}

\begin{center}
\includegraphics[width=\textwidth]{diagrams/system-architecture.png}
\end{center}

\subsection{データモデル(ER図)}

\begin{center}
\includegraphics[width=\textwidth]{diagrams/entity-relationship.png}
\end{center}

\newpage

% 業務フロー
\section{業務フロー}

\subsection{リース発注フロー}

\begin{center}
\includegraphics[width=0.8\textwidth]{diagrams/lease-flow.png}
\end{center}

\newpage

\subsection{安全管理フロー}

\begin{center}
\includegraphics[width=0.85\textwidth]{diagrams/safety-flow.png}
\end{center}

\newpage

% 市場分析
\section{市場分析}

\subsection{建設業界の現状}

\begin{highlight}[title=市場規模]
\begin{itemize}[leftmargin=*]
\item 建設市場全体: 約62兆円(2024年)
\item 建設業許可業者数: 48.4万社
\item 建設業就業者数: 477万人
\end{itemize}
\end{highlight}

\subsection{業界課題}

\begin{enumerate}
\item \textbf{2024年問題への対応} - 時間外労働上限規制の適用開始
\item \textbf{労働力不足} - ピーク時比30\%減、高齢化進行
\item \textbf{DX化の遅れ} - 紙ベースの書類管理が主流
\item \textbf{法規制の強化} - 安全衛生法改正への対応
\end{enumerate}

\newpage

% 競合分析
\section{競合分析}

\subsection{主要競合サービス}

\begin{center}
\begin{tabular}{|l|c|c|c|c|}
\hline
\textbf{サービス} & \textbf{初期費用} & \textbf{月額} & \textbf{導入社数} & \textbf{特徴} \\
\hline
グリーンサイト & 〜30万円 & 〜6万円 & 10万社 & 業界標準 \\
ANDPAD & 10万円 & 3.6万円〜 & 21万社 & 総合管理 \\
Buildee & 要問合せ & 高価格 & 1.7万件 & 大手向け \\
現場Plus & 1万円 & 1万円〜 & 6万社 & 中小向け \\
\hline
\textbf{当社} & \textbf{0円} & \textbf{9,800円〜} & - & \textbf{統合型} \\
\hline
\end{tabular}
\end{center}

\subsection{差別化ポイント}

\begin{tcolorbox}[colback=success!10,colframe=success]
\begin{enumerate}
\item \textbf{8サブシステム統合} - 競合は2-3機能のみ
\item \textbf{ノーコードカスタマイズ} - Lark Base基盤で柔軟対応
\item \textbf{導入スピード} - 最短2週間(競合は1-3ヶ月)
\item \textbf{コストパフォーマンス} - 競合の50\%以下
\end{enumerate}
\end{tcolorbox}

\newpage

% 価格戦略
\section{価格戦略}

\subsection{料金プラン}

\begin{center}
\begin{tabular}{|l|c|c|c|l|}
\hline
\textbf{プラン} & \textbf{月額} & \textbf{従業員数} & \textbf{現場数} & \textbf{推奨顧客} \\
\hline
ライト & 9,800円 & 30名まで & 5件まで & 専門工事業者 \\
スタンダード & 29,800円 & 100名まで & 20件まで & 中小建設会社 \\
プロフェッショナル & 59,800円 & 無制限 & 無制限 & 中堅建設会社 \\
\hline
\end{tabular}
\end{center}

\subsection{競合比較}

\begin{itemize}[leftmargin=*]
\item グリーンサイト比: \textcolor{success}{\textbf{77\%削減}}
\item ANDPAD比: \textcolor{success}{\textbf{50\%削減}}
\end{itemize}

\newpage

% Go-to-Market戦略
\section{Go-to-Market戦略}

\subsection{フェーズ別展開}

\begin{center}
\begin{tabular}{|l|l|l|l|}
\hline
\textbf{フェーズ} & \textbf{期間} & \textbf{目標} & \textbf{主要施策} \\
\hline
I. PMF検証 & 0-6ヶ月 & 100社 & パイロット・紹介 \\
II. 初期成長 & 7-12ヶ月 & 300社 & デジタルマーケ \\
III. スケール & 13-24ヶ月 & 1,200社 & パートナー展開 \\
IV. 市場リーダー & 25-36ヶ月 & 3,000社 & ブランド確立 \\
\hline
\end{tabular}
\end{center}

\subsection{販売チャネル}

\begin{enumerate}
\item \textbf{直販}(60\%)- インバウンド中心
\item \textbf{パートナー}(30\%)- 税理士・社労士・建材商社
\item \textbf{協会連携}(10\%)- 建設業協会セミナー
\end{enumerate}

\newpage

% 収益予測
\section{収益予測}

\subsection{3年収益モデル}

\begin{center}
\begin{tabular}{|l|r|r|r|}
\hline
\textbf{指標} & \textbf{1年目} & \textbf{2年目} & \textbf{3年目} \\
\hline
新規獲得(社) & 300 & 900 & 1,800 \\
累計企業(社) & 300 & 1,200 & 3,000 \\
解約率 & 3\% & 2\% & 1.5\% \\
\hline
ARR & 9,000万円 & 4.3億円 & 11.5億円 \\
売上総利益 & 6,300万円 & 3.4億円 & 9.2億円 \\
営業利益 & -2,700万円 & 2,100万円 & 2.3億円 \\
\hline
\end{tabular}
\end{center}

\subsection{ユニットエコノミクス}

\begin{kpi}
\begin{tabular}{ll}
\textbf{LTV} & 720,000円(平均単価30,000円 × 24ヶ月) \\
\textbf{CAC}(3年目) & 22,000円 \\
\textbf{LTV/CAC} & \textcolor{success}{\textbf{33倍}} \\
\textbf{回収期間} & \textcolor{success}{\textbf{1.1ヶ月}} \\
\end{tabular}
\end{kpi}

\newpage

% 推奨アクションプラン
\section{推奨アクションプラン}

\subsection{Top 3 アクション}

\begin{tcolorbox}[colback=primary!10,colframe=primary,title=1. パイロット顧客10社の確保(即座に実行)]
\begin{itemize}[leftmargin=*]
\item \textbf{Why}: PMF検証と導入事例作成が最優先
\item \textbf{What}: 既存ネットワークから無料トライアル提供
\item \textbf{When}: 今週中に候補リスト作成、来月までに10社確保
\end{itemize}
\end{tcolorbox}

\begin{tcolorbox}[colback=primary!10,colframe=primary,title=2. 製品LP公開とマーケティング開始(1ヶ月以内)]
\begin{itemize}[leftmargin=*]
\item \textbf{Why}: 問い合わせ獲得の仕組み構築が急務
\item \textbf{What}: LP作成、SEO対策、リスティング広告開始
\item \textbf{When}: 2週間でLP公開、4週間で広告運用開始
\end{itemize}
\end{tcolorbox}

\begin{tcolorbox}[colback=primary!10,colframe=primary,title=3. 建設業協会との連携構築(3ヶ月以内)]
\begin{itemize}[leftmargin=*]
\item \textbf{Why}: 中小企業へのリーチに最も効率的
\item \textbf{What}: 地域建設業協会へのアプローチ、セミナー共催
\item \textbf{When}: 1ヶ月で5協会へ接触、3ヶ月で初回セミナー開催
\end{itemize}
\end{tcolorbox}

\newpage

% まとめ
\section{まとめ}

\subsection{投資対効果}

\begin{center}
\begin{tabular}{|l|l|}
\hline
\textbf{指標} & \textbf{値} \\
\hline
LTV/CAC比率 & 33倍(3年目) \\
回収期間 & 1.1ヶ月 \\
損益分岐点 & 470社(2年目Q2達成予想) \\
想定Exit(5年目) & ARR 30億円、バリュエーション100-250億円 \\
\hline
\end{tabular}
\end{center}

\subsection{成功要因}

\begin{enumerate}
\item \textbf{市場タイミング} - 2024年問題への対応需要増
\item \textbf{価格優位性} - 競合の50\%以下のコスト
\item \textbf{統合ソリューション} - 8サブシステム一括提供
\item \textbf{導入容易性} - 最短2週間で運用開始
\item \textbf{カスタマイズ性} - Lark Baseの柔軟性
\end{enumerate}

\vspace{1cm}

\begin{center}
\begin{tcolorbox}[colback=success!20,colframe=success,width=10cm]
\centering
\Large\bfseries
本事業計画に基づき\\
早期の市場参入を推奨します
\end{tcolorbox}
\end{center}

\end{document}
